
%% Masterthesis
%% 2018
%%%%%%%====================== Preamble

%%%======================== Main.tex

%% Schriftgr��e, Layout, Papierformat, Art des Dokumentes
\documentclass[11pt, oneside, openright, a4paper, headings=normal,
abstracton, bibliography=totoc, listof=totoc]{scrreprt}
% Schriftgroesse: 11pt
% Layout: Einseitig
% twoside Doppelseitig
% Papierformat: Din A4

%% Neue deutsche Rechtschreibung
\usepackage[ngerman]{babel}

%% Dokumentencodierung - wichtig um Sonderzeichen direkt im Dokument zu nutzen
\usepackage[latin1]{inputenc}
% Mac - applemac
% Linux - utf8
% Windows - latin1

%% Verwendung von Schrifen, die Umlaute enthalten
\usepackage[T1]{fontenc}
\usepackage{lmodern}

%% Literaturverzeichnis
\usepackage[square,sort,comma,numbers]{natbib}
%	\setcitestyle{square}
	%, comma, numbers,sort&compress, super}
% Literaturverzeichnis nach Deutscher DIN Norm


%% Name des Literaturverzeichnisses anpassen
\renewcommand{\refname}{Literaturverzeichnis}


%% Definieren von "Abk�rzungen"
\newcommand{\Autor}{Senorping}
\newcommand{\Adresse}{743 Evergreenterace}
\newcommand{\Mail}{alias@domain.com}
\newcommand{\Arbeit}{Master-Thesis}
\newcommand{\ArbeitThema}{Enchantin muggel technology} %Titel
\newcommand{\Prof}{Prof. Dr.-Ing. Sirius Snape}



% f�r halbes Leerzeichen bei diesen Abk�rzungen und bei Einheiten
\newcommand{\zB}{\mbox{z.\,B. }} %z.B. zum Beispiel
\newcommand{\uA}{\mbox{u.\,A. }} %u.A. unter Anderem

%% SI EInheiten verwenden
\usepackage{siunitx}

%% Einstellungen der Seitenraender
%\usepackage[left=3cm,right=3cm,top=2.5cm,bottom=2.5cm]{geometry}

%% Fuer Rotationen
\usepackage{rotating}

%% eigene Kopf- und Fu�zeilen
%\usepackage{fancyhdr}
%\lhead[\fancyplain{}{\bfseries \small{\rightmark}}]%
%      {\fancyplain{}{\bfseries \small{}}}
%\rhead[\fancyplain{}{\bfseries \small{}}]%
%      {\fancyplain{}{\bfseries \small{\rightmark}}}
%\lfoot[\fancyplain{}{\bfseries \small{\thepage}}]%
%      {\fancyplain{}{\bfseries \small{\leftmark}}}
%\rfoot[\fancyplain{}{\bfseries \small{\leftmark}}]%
%      {\fancyplain{}{\bfseries \small{\thepage}}}
%\cfoot{}

%% Inhaltsverzeichnis - enthaelt auch bei Kapiteln Punkte zur Seite
\usepackage{tocloft}
%\addtocontents{toc}
%{\protect\renewcommand{\protect\cftsecleader}
%{\protect\cftdotfill{\protect\cftsubsecdotsep}}{}}
%\setcounter{tocdepth}{4}
%\setcounter{secnumdepth}{4}

%% Zeilenabstand
\usepackage{setspace}
\onehalfspacing
% 1.5 facher Zeilenabstand


%% Mathematische Pakete
\usepackage{amsmath}        %Gleichungsumgebungen
\usepackage{arrayjobx}		%Umgeht Fehler by \matr Befehl!
\usepackage{amssymb}        %erweiterter Symbolsatz
\usepackage{latexsym}       %zusaetzliche Symbole
\usepackage{exscale}        %passt sonderzeichen wie Integrale der Schriftgroesse an
\usepackage{physics}

%% Zusammenfassung
\usepackage{abstract}

%% Um Bilder einzubinden

\usepackage{graphicx}
%\DeclareGraphicsExtensions{.pdf} % Dateiendung f�r PDF-Bilder
\usepackage{float} %um Bilder mit H besser zu positionieren

%% Farbenunterstuetzung im Dokument
\usepackage{color}

%% kleinere Bildunterschrift
%\usepackage[hang,footnotesize,bf]{caption}
\usepackage[font=small,format=plain,labelfont=bf,up]{caption}
%,textfont=it,up

%% Sonderzeichen
\usepackage{textcomp}

%% Um Quellcode einzubetten
\usepackage{listings}
\lstset{
	numbers=left, 
	numberstyle=\tiny, 
	numbersep=5pt,
	commentstyle=\color[rgb]{0.0,0.4,0.0},
	keywordstyle=\color{black}\bfseries,
	extendedchars=true,
	basicstyle=\scriptsize\ttfamily,
	basicstyle=\footnotesize\ttfamily,
	tabsize=4,
	breakautoindent=true,
	breakindent=2em,
	breaklines=true,
	%caption={Codebeispiel},
	captionpos=b,
	%frame=tlrb,
	%backgroundcolor=\color[rgb]{0.0,0.0,1.0},
	%prebreak=\mbox{$\hookleftarrow$},
	language=octave
}

%% Internes Paket um zu ueberpruefen ob PDFLaTeX verwendet wird
\usepackage{ifpdf}


%% Wenn es eine PDFLaTeX ist wird automatisch die Dateiendung der Dateien gesetzet
\ifpdf
  \DeclareGraphicsExtensions{.pdf, .jpg, .tif}
\else
  \DeclareGraphicsExtensions{.eps, .jpg}
\fi

%% Kapitel auf neuer Seite Anfangen
%\newcommand{\origsection}{} 
%\let\origsection\section 
%\renewcommand*{\section}{% 
% \clearpage 
%  \origsection 
%} 


%%%%%%%%%%%%%%%%%%%%%%%%%%%%%%%%%%%%%%%%%%%%%%%%%%%%%%%%%%%%%%%%%%%%%%%%%%%%%%%%%%%%%%%%%%%%%%%%%%%%%%%%%%%%%%%%%%
%%%%% Verzeichnisse %%%%%%%%%%%%%% Abk�rzung, Symbol, Indices   %%%%%%%%%%%%%%%%%%%%%%%%%%%%%%%%%%%%%%%%%%%%%%%%%%
%%%%%%%%%%%%%%%%%%%%%%%%%%%%%%%%%%%%%%%%%%%%%%%%%%%%%%%%%%%%%%%%%%%%%%%%%%%%%%%%%%%%%%%%%%%%%%%%%%%%%%%%%%%%%%%%%%


%%%%%%%%%%%%%%%%%%%%%%%%%%%%%%%%%%%%%%%%%%%%%%%%%%%%%%%%%
%%% Glossaries package einbinden %%%%%%%%%%%%%%%%%%%%%%%%
%%%%%%%%%%%%%%%%%%%%%%%%%%%%%%%%%%%%%%%%%%%%%%%%%%%%%%%%%

\usepackage[
nonumberlist, 		%keine Seitenzahlen anzeigen
acronym,      		%ein Abk�rzungsverzeichnis erstellen
toc,          		%Eintr�ge im Inhaltsverzeichnis
section=chapter,	%im Inhaltsverzeichnis auf chapter-Ebene erscheinen
xindy
]{glossaries} 

%% Den Punkt am Ende jeder Beschreibung deaktivieren, da kein Satz
\renewcommand*{\glspostdescription}{}

%% Tabellenbreite & L�nge \hline in Verzeichnissen
\setlength{\glsdescwidth}{16cm}


%%%%%%%%%%%%%%%%%%%%%%%%%%%%%%%%%%%%%%%%%%%%%%%%%%%%%%%%%
%%% Zus�tzliche Verzeichnisse anlegen %%%%%%%%%%%%%%%%%%%
%%%%%%%%%%%%%%%%%%%%%%%%%%%%%%%%%%%%%%%%%%%%%%%%%%%%%%%%%


%%%%%%%%%%%%%%%%%%%%%%%%%%%%%%%%%%%%%%%%%%%%%%%%%%%%%%%%%
%%% Symbolverzeichnis %%%%%%%%%%%%%%%%%%%%%%%%%%%%%%%%%%%
%%%%%%%%%%%%%%%%%%%%%%%%%%%%%%%%%%%%%%%%%%%%%%%%%%%%%%%%%

%% Symbolverzeichnis erstellen == Griechisch & R�misch =======================

\newglossary[slg]{romsymbols}{syi}{syg}{R�mische Buchstaben} %% R�mische Symbole
\newglossary[slg]{grksymbols}{sgi}{sgg}{Griechische Buchstaben} %% Griechische Symbole
% Symbolverzeichnis um SI-Einheiten erweitern
\glsaddkey{unit}{\glsentrytext{\glslabel}}{\glsentryunit}{\GLsentryunit}{\glsunit}{\Glsunit}{\GLSunit}

\glsnoexpandfields

%% Ausgabe des Symbolverzeichnis formatieren, neuen Style definieren der Einheit ausgibt
\newglossarystyle{symbunitlong}{%
	\setglossarystyle{long3col}% base this style on the list style
	\renewenvironment{theglossary}{% Change the table type --> 3 columns
		\begin{longtable}{cp{0.6\glsdescwidth}>{\centering\arraybackslash}p{3cm}}}%
		{\end{longtable}}%
	%
	\renewcommand*{\glossaryheader}{%  Change the table header
		\bfseries  & \bfseries Definition & \bfseries Einheit \\
		\hline
		\endhead}
	\renewcommand*{\glossentry}[2]{%  Change the displayed items
		%\begin{center}
		\glstarget{##1}{\glossentryname{##1}} %
		%\end{center}
		& \glossentrydesc{##1}% Description
		& \glsunit{##1}  \tabularnewline
	}
}


%%%%%%%%%%%%%%%%%%%%%%%%%%%%%%%%%%%%%%%%%%%%%%%%%%%%%%%%%
%%% Indices %%%%%%%%%%%%%%%%%%%%%%%%%%%%%%%%%%%%%%%%%%%%%
%%%%%%%%%%%%%%%%%%%%%%%%%%%%%%%%%%%%%%%%%%%%%%%%%%%%%%%%%


%% Verzeichnis f�r Indices anlegen =============================================
\newglossary[ilg]{indiceslist}{idi}{idg}{Indizes}

%% Ausgabe des Indexverzeichnis & Abk�rzungsverzeichnis formatieren --> neuen Style definieren
\newglossarystyle{indiclong}{%
	\setglossarystyle{long3col}% base this style on the list style
	\renewenvironment{theglossary}{% Change the table type --> 3 columns
		\begin{longtable}{cp{0.6\glsdescwidth}>{\centering\arraybackslash}p{2cm}}}%
		{\end{longtable}}%
	%
	\renewcommand*{\glossaryheader}{%  Change the table header
		\bfseries  & \bfseries Definiton\\
		\hline
		\endhead}
	\renewcommand*{\glossentry}[2]{%  Change the displayed items
		%\begin{center}
		\glstarget{##1}{\glossentryname{##1}} %
		%\end{center}
		& \glossentrydesc{##1}% Description
		\tabularnewline
	}
}

%%%%%%%%%%%%%%%%%%%%%%%%%%%%%%%%%%%%%%%%%%%%%%%%%%%%%%%%%%%%%%%%%%%
%%% In TeX-Dateien ausgelagerte  Glossaries Inhalte einbinden %%%%%
%%%%%%%%%%%%%%%%%%%%%%%%%%%%%%%%%%%%%%%%%%%%%%%%%%%%%%%%%%%%%%%%%%%

%%%% Abkürzungen die ins Abkürzungsverzeichnis aufgenommen werden soll

\newacronym{acr:ibd}{ibd.}{Ibidem, ebenda, ebendort}

\newacronym{CD}{CD}{Compact Disc}

\newacronym{acr:akm}{AKM}{Aktivit\"atskoeffizienten-Modell} % \protect\glsadd{glos:AKM}



%%% Auslagerungsdatei verwendeter griechischer Symbole
%%% Identifier romsymbols

%%%%%%%%%%%%%%%%%%%%%%%%%%%%%%%%%%%%%%%%%%%%%%%%%%%%%%%%%%%%%%%%
%% R�mische Symbole %%%%%%%%%%%%%%%%%%%%%%%%%%%%%%%%%%%%%%%%%%%%
%%%%%%%%%%%%%%%%%%%%%%%%%%%%%%%%%%%%%%%%%%%%%%%%%%%%%%%%%%%%%%%%

\newglossaryentry{symb:abschargeelec}{
name=$\abs{z_cz_a}$,
description={Absolutes Ladungsprodukt eines Elektrolyten},
unit={[-]},
sort=zzzabsolute,
type=romsymbols
}

\newglossaryentry{symb:elecstatforce}{
name=$K$,
description={Elektrostatische Anziehungkraft zwischen Punktladungen},
unit={\si{N}},
sort=kkkelectrostatic,
type=romsymbols
}

\newglossaryentry{symb:pointcharge}{
name=$q_i$,
description={Punktladung},
unit={\si{C}},
sort=qqqpointcharge,
type=romsymbols
}

\newglossaryentry{symb:radialdist}{
name=$r$,
description={Radialer Abstand},
unit={\si{m}},
sort=rrrradial,
type=romsymbols
}


\newglossaryentry{symb:ionwert}{
name=$z_i$,
description={Ionen-Wertigkeit des Spezies i},
unit={[-]},
sort=wertigkeit,
type=romsymbols
}

\newglossaryentry{symb:massfrac}{
name=$w_i$,
description={Massen-Fraktion der Spezies i},
unit={[-]},
sort=weightfrac,
type=romsymbols
}

\newglossaryentry{symb:molmass}{
name=$M_i$,
description={Molare Masse der Spezies i},
unit={[-]},
sort=molarmass,
type=romsymbols
}

\newglossaryentry{symb:molarity}{
name=$c_i$,
description={Molarit�t der Spezies i},
unit={\si{mol.m^{-3}}},
sort=molarity,
type=romsymbols
}

\newglossaryentry{symb:molality}{
name=$m_i$,
description={Molalit�t der Spezies i},
unit={\si{mol.kg^{-1}}},
sort=molality,
type=romsymbols
}

\newglossaryentry{symb:aktivity}{
name=$a_i$,
description={Ionen-Konzentration der Spezies i},
unit={\si{mol.kg^{-1}}},
sort=activity,
type=romsymbols
}

\newglossaryentry{symb:ionstr}{
name=$I_m$,
description={Mittlere molale Ionenst�rke},
unit={\si{mol.kg^{-1}}},
sort=ionstrength,
type=romsymbols
}

\newglossaryentry{symb:debhueckconst}{
name=$A_m$,
description={Debye-H�ckel Konstante, Temperatur- und L�sungsmittelabh�ngig},
unit={\si{mol^{0.5}.kg^{-0.5}}},
sort=debbyhueckel,
type=romsymbols
}

\newglossaryentry{symb:kboltzmann}{
name=$k_B$,
description={Boltzmannkonstante},
unit={\si{J.K^{-1}}},
sort=kboltzmann,
type=romsymbols
}

\newglossaryentry{symb:avrogado}{
name=$N_A$,
description={Avrogadokonstante \num{6.022e26}},
unit={\si{mol^{-1}}},
sort=navrogado,
type=romsymbols
}

\newglossaryentry{symb:bpara}{
name=$B_i$,
description={Einzelionenbeitrag von Anion oder Kation, additiver Anteil Bromley-Theorie},
unit={[-]},
sort=bbbpara,
type=romsymbols
}

\newglossaryentry{symb:Fpara}{
name=$F_i$,
description={Einzelionenbeitrag von Anion oder Kation, additiver Anteil Bromley-Theorie},
unit={[-]},
sort=fffpara,
type=romsymbols
}

\newglossaryentry{symb:elemencharge}{
name=$e$,
description={Elementarladung des Elektrons \num{-1.6022e-19}},
unit={\si{C}},
sort=fffpara,
type=romsymbols
}
%%% Auslagerungsdatei verwendeter griechischer Symbole
%%% Identifier grksymbols

%%%%%%%%%%%%%%%%%%%%%%%%%%%%%%%%%%%%%%%%%%%%%%%%%%%%%%%%%%%%%%%%
%% Griechische Symbole %%%%%%%%%%%%%%%%%%%%%%%%%%%%%%%%%%%%%%%%%
%%%%%%%%%%%%%%%%%%%%%%%%%%%%%%%%%%%%%%%%%%%%%%%%%%%%%%%%%%%%%%%%

\newglossaryentry{symb:rho}{
name=$\varrho$,
description={Dichte},
unit={\si{kg.m^{-3}}},
sort=symbolrho,
type=grksymbols
}

\newglossaryentry{symb:gammam}{
name=$\gamma_i$,
description={Aktivit�tskoeffizient der Spezies i},
unit={[-]},
sort=symbolgamma,
type=grksymbols
}

\newglossaryentry{symb:epsilon0}{
name=$\epsilon_0$,
description={Elektrische Feldkonstante \num{8.854e-12}},
unit={\si{C^2.N^{-1}.m^{-2}}},
sort=symbolepsilon0,
type=grksymbols
}

\newglossaryentry{symb:epsilont}{
name=$\epsilon(T)$,
description={Temperaturabh�ngige relative Dielektrizit�tskonstante},
unit={[-]},
sort=symbolepsilonoft,
type=grksymbols
}

\newglossaryentry{symb:epsilon}{
name=$\epsilon$,
description={Permittivi�t},
unit={\si{C^2.N^{-1}.m^{-2}}},
sort=symbolepsilon,
type=grksymbols
}

\newglossaryentry{symb:beta}{
name=$\beta$,
description={Stoffpezifischer Temperaturunabh�ngiger Paramter (Debye-H�ckel Theorie)},
unit={[-]},
sort=symbolbeta,
type=grksymbols
}

\newglossaryentry{symb:delta}{
name=$\delta$,
description={Einzelionenbeitrag, multiplikativer Anteil, Bromley-Theorie},
unit={[-]},
sort=symboldelta,
type=grksymbols
}
%Befehle für Glossar
\newglossaryentry{glos:AD}
{name=Active Directory,
description={Active Directory ist in einem Windows 2000/" "Windows
Server 2003-Netzwerk der Verzeichnisdienst, der die zentrale \ldots.}
}

\newglossaryentry{glos:AntwD}{
name=Antwortdatei,
description={Informationen zum Installieren einer Anwendung oder des Betriebssystems.}
}
\newglossaryentry{ind:lm}{
	name=$LM$,
	description={L�sungsmittel},
	sort=loesung,
	type=indiceslist
}

\newglossaryentry{ind:fl}{
	name=$fl$,
	description={Fluide Phase},
	sort=fluid,
	type=indiceslist
}

\newglossaryentry{ind:anion}{
	name=$a$,
	description={Anion},
	sort=anion,
	type=indiceslist
}

\newglossaryentry{ind:cation}{
	name=$c$,
	description={Kation},
	sort=cation,
	type=indiceslist
}

\newglossaryentry{ind:anicati}{
	name=$ca$,
	description={Anion-Kation Interaktion},
	sort=cationanion,
	type=indiceslist
}

\newglossaryentry{ind:debhueckactcoeff}{
	name=$\pm$,
	description={Aktivit�tskoeffizient innerhalb der Debye-H�ckel Theorie},
	sort=zzzzpm,
	type=indiceslist
}

%%%%%%%%%%%%%%%%%%%%%%%%%%%%%%%%%%%%%%%%%%%%%%%%%%%%%%%%%%%%%%%%%%%
%%% Glossar-Befehle anschalten %%%%%%%%%%%%%%%%%%%%%%%%%%%%%%%%%%%%
%%%%%%%%%%%%%%%%%%%%%%%%%%%%%%%%%%%%%%%%%%%%%%%%%%%%%%%%%%%%%%%%%%%
\makeglossaries


%%%%%%%%%%%%%%%%%%%%%%%%%%%%%%%%%%%%%%%%%%%%%%%%%%%%%%%%%%%%%%%%%%%%%%%%%%%%%%%%%%%%%
%%% PDF Navigation im Dokument %%%%%%%%%%%%%%%%%%%%%%%%%%%%%%%%%%%%%%%%%%%%%%%%%%%%%%
%%%%%%%%%%%%%%%%%%%%%%%%%%%%%%%%%%%%%%%%%%%%%%%%%%%%%%%%%%%%%%%%%%%%%%%%%%%%%%%%%%%%%

\usepackage[pdfview=FitH,     	
            pdfstartview=FitH,
            pageanchor=true,
            colorlinks=true,
            linkcolor=black,
            citecolor=black,
            menucolor=black,
            urlcolor=black,
            breaklinks=true,
            plainpages=false
           ]{hyperref}
%% Silbentrennung unterbinden
% Befehl: \nohyphens{ZUSAMMENGESCHRIEBENESwort}
\usepackage{hyphenat}

% Farbe
\usepackage{color}

% Mehrere Zeilen in einer Tabelle
\usepackage{multirow}

% Define user colors using the RGB model
\definecolor{hellrot}{rgb}  {0.95,0.50,0.50}
\definecolor{hellgruen}{rgb}{0.50,0.95,0.50}
\definecolor{hellgelb}{rgb} {0.95,0.95,0.50}

% Farbe in Tabellen
\usepackage{colortbl}

%% Neue Befehle
%\newcommand{\dd}{\mathrm{d}}


%%%%%%%%%%%%%%%%%%%%%%%%%%%%%%%%%%%%%%%%%%%%%%%%%%%%%%%%%%%%%%%%%%%%%%%%%%%%%%%%%%%%%
%%% Document Anfang %%%%%%%%%%%%%%%%%%%%%%%%%%%%%%%%%%%%%%%%%%%%%%%%%%%%%%%%%%%%%%%%%
%%%%%%%%%%%%%%%%%%%%%%%%%%%%%%%%%%%%%%%%%%%%%%%%%%%%%%%%%%%%%%%%%%%%%%%%%%%%%%%%%%%%%

\begin{document}


% Roemische Seitenzahlen
\pagenumbering{Roman}

\lstset{
	numbers=left, 
	numberstyle=\tiny, 
	numbersep=5pt,
	%commentstyle=\color[rgb]{0.0,0.4,0.0},
	keywordstyle=\color{black}\bfseries,
	extendedchars=true,
	basicstyle=\scriptsize\ttfamily,
	basicstyle=\footnotesize\ttfamily,
	tabsize=4,
	breakautoindent=true,
	breakindent=2em,
	breaklines=true,
	%caption={Codebeispiel},
	captionpos=b,
	%frame=tlrb,
	%backgroundcolor=\color[rgb]{0.0,0.0,1.0},
	%prebreak=\mbox{$\hookleftarrow$},
	language=octave
}

%%%
%%%================================================================== Title
%%
%% Titelblatt
%%

\begin{titlepage}

% Logo der Hochschule und des ext Unternehmens
\parbox[t]{\textwidth}{
	\parbox{0.65\textwidth}{
		%\parbox[c]{2.0cm}{
		%	\vspace{-1.0cm}
		%	\hspace*{0.2cm}
			\includegraphics[width=4cm]{img/title/logo_TH-Koeln_CMYK_22pt.eps}	
		%}
	}
	\parbox{0pt}{
		%\parbox[c]{2.0cm}{
		%	\vspace{-1.0cm}
		%	\hspace*{0.2cm}
			%\includegraphics[width=5cm]{img/title/logo-scai.pdf}	
		%}		
	}
}

\begin{center}
	\vspace{3cm}
	\LARGE\textbf{\Arbeit}\\
	\vspace{0.5cm}
	\normalsize	in Grad-School \\Hogwartian Magic School\\
	\vspace{1cm}
	\Large\textbf{\ArbeitThema}\\
	\vspace{0.5cm}
	\normalsize\textbf{von}\\
	\vspace{0.2cm}
	\textbf{\Autor}\\
	\Mail
\end{center}

\vfill
\begin{center}
\begin{small}
\parbox{0cm}{
	\begin{tabbing}
		 \hspace{3.5cm}\=\kill %??? Was macht \kill ?
		 Projektbetreuer:\> \Prof\\ %??? \> ist ein Tab
		 \\
		 Eingereicht am:\> \today\\
	\end{tabbing}
	}
\end{small}
\end{center}

\end{titlepage}

%%%
%%%================================================================== Abstract

% Neue leere Seite
\mbox{}\thispagestyle{empty}
%\clearpage

% Ueberschrift soll im Inhaltsverzeichnis erscheinen
%\addcontentsline{toc}{section}{Zusammenfassung}

%\include{00b_abstract}

%%%
%%%================================================================== Statement

% Neue Seite
\clearpage

% Ueberschrift soll im Inhaltsverzeichnis erscheinen
%\addcontentsline{toc}{section}{Erkl�rung}

\begin{center}\textbf{\large{Erkl\"arung}}\end{center}

\noindent
Ich versichere an Eides statt, die von mir vorgelegte Arbeit selbstst�ndig verfasst zu haben. Alle Stellen, die w�rtlich oder sinngem�� aus ver�ffentlichten oder nicht ver�ffentlichten Arbeiten anderer entnommen sind, habe ich als entnommen kenntlich gemacht. S�mtliche Quellen und Hilfsmittel, die ich f�r die Arbeit benutzt habe, sind angegeben. Die Arbeit hat mit gleichem Inhalt bzw. in wesentlichen Teilen noch keiner anderen Pr�fungsbeh�rde vorgelegen.
 Die aus fremden Quellen direkt oder indirekt �bernommenen Gedanken sind als solche kenntlich gemacht. Die Arbeit wurde bisher in gleicher oder �hnlicher Form noch keiner anderen Pr�fungsbeh�rde vorgelegt bzw. nicht ver�ffentlicht.
 
\vspace{4\baselineskip}

\begin{tabular}{cp{10cm}}
 \rule[-0.2cm]{7.5cm}{0.5pt} \\
 \textsc{\Autor} \\
 \begin{scriptsize}\Adresse\end{scriptsize}
\end{tabular}

%%%


%%%================================================================== TOC

\clearpage
\tableofcontents


%%%================================================================== Dummy Inhalte Glossaries
\glsaddall % F�ge alle Eintr�ge, egal ob genutzt ein!


%%%================================================================== Abkuerzungsverzeichnis
\clearpage
%Abk�rzungen ausgeben
%\deftranslation[to=German]{Acronyms}
\printglossary[type=\acronymtype,style=indiclong, title={Abk�rzungsverzeichnis}]


%%%================================================================== Symbolverzeichnis
\clearpage
%Symbole ausgeben
\printglossary[type=romsymbols,style=symbunitlong]
\printglossary[type=grksymbols,style=symbunitlong]


%%%================================================================== Indices
\clearpage
%Symbole ausgeben
\printglossary[type=indiceslist,style=indiclong]


%%%
%%%================================================================== Content

% Neue Seite
\clearpage

% Arabische Seitenzahlen
\pagenumbering{arabic}

%% Kapitel der Arbeit
%\pagestyle{fancy}

%%
%% 1 - Einleitung
%%

\chapter{Einleitung}

Einleitungstext\\
Hinweis: Das Dokument ist für den Duplex-Druck (zweiseitig) bestimmt und hat deshalb viele Leerseiten.

%%
%% 2 - Grundlagen
%%
\chapter{Theoretische Grundlagen}

In diesem Kapitel werden die Theorien und Grundlagen die zur Bearbeitung der angefertigten Arbeit beigetragen haben. Darunter handelt es sich um direkt oder indirekt tangierte Arbeiten und Erkenntnisse aus unterschiedlichster wissenschaftlicher und nicht wissenschaftlichen Publikationen des 20. und 21. Jahrhunderts. Haupts�chlich konzentrieren sich die Themenfelder auf die folgenden Bereiche

\begin{itemize}
\item{lorem loren et funghum sporem}
\label{item:itemtitle}
\end{itemize}

Lorem ipsum dolor sit amet, consetetur sadipscing elitr, sed diam nonumy eirmod tempor invidunt ut labore et dolore magna aliquyam erat, sed diam voluptua. At vero eos et accusam et justo duo dolores et ea rebum. Stet clita kasd gubergren, no sea takimata sanctus est Lorem ipsum dolor sit amet. Lorem ipsum dolor sit amet, consetetur sadipscing elitr, sed diam nonumy eirmod tempor invidunt ut labore et dolore magna aliquyam erat, sed diam voluptua. At vero eos et accusam et justo duo dolores et ea rebum. Stet clita kasd gubergren, no sea takimata sanctus est Lorem ipsum dolor sit amet., vgl. \cite{ref:noauthor_wertigkeit_1998, ref:luckas_thermodynamik_2013}.

\begin{equation}
z_{H^+} = 1 , z_{Fe^{3+}} = 3 , z_{Cl^-} = -1 , z_{SO_4^{-2}} = -2
\label{eq:wertigkeit}
\end{equation}

Lorem ipsum dolor sit amet, consetetur sadipscing elitr, sed diam nonumy eirmod tempor invidunt ut labore et dolore magna aliquyam erat, sed diam voluptua. At vero eos et accusam et justo duo dolores et ea rebum. Stet clita kasd gubergren, no sea takimata sanctus est Lorem ipsum dolor sit amet. Lorem ipsum dolor sit amet, consetetur sadipscing elitr, sed diam nonumy eirmod tempor invidunt ut labore et dolore magna aliquyam erat, sed diam voluptua. At vero eos et accusam et justo duo dolores et ea rebum. Stet clita kasd gubergren, no sea takimata sanctus est Lorem ipsum dolor sit amet., wie in \eqref{eq:ionstrength} dargestellt, vgl. \cite{ref:luckas_thermodynamik_2013}:

\begin{equation}
I_m = \frac{1}{2} \sum_i m_i z_i^2
\label{eq:ionstrength}
\end{equation}

\subsection{Thermodynamik von Elektrolytl�sungen}
Lorem ipsum dolor sit amet, consetetur sadipscing elitr, sed diam nonumy eirmod tempor invidunt ut labore et dolore magna aliquyam erat, sed diam voluptua. At vero eos et accusam et justo duo dolores et ea rebum. Stet clita kasd gubergren, no sea takimata sanctus est Lorem ipsum dolor sit amet. Lorem ipsum dolor sit amet, consetetur sadipscing elitr, sed diam nonumy eirmod tempor invidunt ut labore et dolore magna aliquyam erat, sed diam voluptua. At vero eos et accusam et justo duo dolores et ea rebum. Stet clita kasd gubergren, no sea takimata sanctus est Lorem ipsum dolor sit amet., vgl. \citep[S. 8]{ref:luckas_thermodynamik_2013}.

\begin{eqnarray}
m_i = \frac{c_i}{\rho_{LM}w_{LM}}\\
w_{LM} = 1- \frac{1}{w_{LM}}\sum c_j\frac{M_j}{1000}
\label{eq:massenbruchLM}
\end{eqnarray}



%%
%% 3 - Konzept
%%

\chapter{Konzept zur Modellbildung}

\section{GNU Octave}
\lstinputlisting{code/liqdens.m}

%%
%% 4 - Implementierung
%%

\chapter{Implementierung des XYZ}


%%
%% 5 - Auswertung
%%

\chapter{Analyse der Ergebnisse}

\section{Komplexitätsanalyse des Programms}

\subsection{Programmabschnitt XYZ}

$\mathcal{O}(n)$

%%
%% 6 - Fazit
%%


\chapter{Zusammenfassung \& Ausblick}

Das Ergebnis der Arbeit ist ...


\thispagestyle{plain}
%%%
%%%================================================================== Glossar

\begin{appendix}

\glstoctrue % Glossar im Inhaltsverzeichnis
%\printglossaries
%[style=altlist,title=Glossar]

%%%
%%%================================================================== ListOf

\clearpage
\listoffigures

\clearpage
\listoftables

\clearpage
\renewcommand{\lstlistlistingname}{Listingverzeichnis}
\lstlistoflistings

%%%
%%%================================================================== Bibliography

% Neue Seite
\clearpage

% Ueberschrift soll im Inhaltsverzeichnis erscheinen
%\addcontentsline{toc}{section}{Literatur}

\bibliographystyle{unsrt}
% Alternative: abbrv, alpha, plain,...

% Verwendet die Textdatenbank literature.bib
% Alter Befehl \biobliography{literature}
\bibliography{literature.bib}

%%%
%%%================================================================== Dokument Ende

\end{appendix}

\end{document}

%%% EOF
